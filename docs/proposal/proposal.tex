\documentclass[12pt]{article}

\usepackage{fullpage} % does reasonable things
\usepackage{setspace} % allows double spacing

\begin{document}
\title{}
\author{Mark Gius}
\date{\today}
\maketitle

\section{Project Description}
I would like to build an RPG style game that requires the player to control
characters through programmable strategies, rather than manual control. This
project is similar to RoboCode (http://robocode.sourceforge.net), a tank
warfare game where player-written AI's duel.

\section{Primary Feature Goals}
A playable JRPG-style game, similar to Super Nintendo era games.  Players
control a number of player characters who are able to battle with computer
controlled characters. The player characters are able to use weapons, cast
spells, and use items.  Control of the player characters is performed
exclusively by sets of condition-response pairs, which I call strategies.  For
example: a condition of health less than 50\% might have an associated response
to heal with a potion.  Players must construct their strategies well to succeed
in the game. Primary focus for the project will be the battle system. GUI
interface using the PyGame development libraries.

\section{Secondary Feature Goals}
Player vs. Player interaction would be a fun addition to the game, but is not a
mandatory requirement.  I would like for every aspect of the game to be easily
pluggable.  For example, adding new spells, weapons, and monsters should be
trivial.  It should also be simple for a motivated coder to replace the
strategies system with either a manually controlled interface, or some other
controlling mechanism.  The ability to create and navigate ``campaigns'' that
the player can navigate through is a relatively import secondary objective.

\section{Special Requests}
I require no special hardware for this project.  It would be helpful if the
PyGame development libraries were installed to at least some on-campus lab
machines to allow development of the program on campus. This software is
already present in the repositories for the department linux machines, but it
is not currently installed.

\end{document}
